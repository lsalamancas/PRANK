\chapter{PROJECT RESOURCES}\label{projects_resources}
\graphicspath{{images/}}
This chapter outlines the technical requirements, material resources, and estimated costs associated with the development of the PRANK system (Parallel Robot for ANKle rehabilitation). It includes a detailed breakdown of mechanical components, electronic subsystems, software tools, and validation instruments necessary to construct and implement the rehabilitation platform. Additionally, the chapter provides an initial estimation of development time and budget, supporting the project's feasibility and helping guide logistical and financial planning. These projections will serve as a reference for resource allocation during the execution phase of the project.

\begin{table}[H]
\centering
\caption{Preliminary Resource and Cost Estimation for PRANK}
\begin{tabular}{|p{4cm}|p{5.5cm}|p{2.5cm}|}
\hline
\rowcolor{Gray}
\textbf{Category} & \textbf{Description} & \textbf{Estimated Cost (USD)} \\
\hline
3D Printing & PLA filament, high-resolution printing (frame and joint parts) & 180 \\
\hline
Electronic Components & Microcontroller (STM32), force sensor, IMUs, wiring, connectors & 500 \\
\hline
Motors & 3 × Brushless DC motors with encoders & 240 \\
\hline
Motor Drivers & Compatible drivers with current control & 90 \\
\hline
Mechanical parts & Bearings, couplings, screws, aluminum parts & 100 \\
\hline
VR Headset & Oculus/Meta Quest 2 (or equivalent) & 300 \\
\hline
Software Licenses & Unity Pro/Unreal (if needed), MATLAB/Simulink (edu license) & 0--100 \\
\hline
Validation Tools & Access to optoelectronic motion capture lab  & 50 \\
\hline
Personal Labor & Estimation of 120 hours × \$10/hr (development + testing) & 1,200 \\
\hline
\textbf{Total Budget} & & \textbf{2,760} \\
\hline
\end{tabular}
\end{table}