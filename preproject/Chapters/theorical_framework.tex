\chapter{THEROTICAL FRAMEWORK}\label{introduction}
\graphicspath{{images/}}

% Importance of ankle in the gait and all the diseases that causes footdrop or similars, and the solutions for this desease.

\section{Human gait}

Human gait is one of the most important tasks of daily life, as it allows people to maintain independence and live with autonomy, mobility, and dignity. Impairments in gait due to neurological or musculoskeletal conditions can significantly affect quality of life, making rehabilitation a critical component of recovery and functional reintegration.

The gait analysis is carried out on the basis of the gait cycle, that can be taken as representation of person's walking patterns, and comparison of several cycles indicates the variability of the pattern \cite{Baker2013}. The gait cycle is normalized from 0\% to 100\%, because each pattern have different timing, so is not objective realize a comparison between them \cite{Medved2021}, it can be described with spatial and temporal parameters. 
\subsection{Spatial parameters}

The main spatial parameters are shown in figure \ref{step_stride} and described bellow:

\begin{itemize}
    \item Step: the movement of the one foot in front of the other.
    \item Stride: step of one foot followed by another step for the other. 
    \item Foot contact: it is considered as the beginning of the gait cycle, in healthy people it is referred to as \textit{heel strike}.  
    \item Step length: distance traveled for one foot in front of the same part of the other foot.
    \item Stride length: distance between two consecutive gait cycles.
    \item Step width: mediolateral separation of the feet, also known as stride width.
\end{itemize}

\begin{figure}
    \centering
    \includegraphics[width=1\textwidth]{step_stride.png}
    \caption[Gait spatial parameters]{Gait spatial parameters. Source: edited by author from \cite{Baker2013}.}
    \label{step_stride}
\end{figure}

\subsection{Temporal parameters}
Temporal parameters are listed bellow:

\begin{itemize}
    \item Stride time: duration of gait cycle (time between two foot strikes). 
    \item Cadence: it is a more commonly term used to specify the duration of the cycle in indirect way. And is described by: 
    \begin{equation}
        \frac{\#cycles}{time\hspace{1mm}interval}
    \end{equation}
    \item Walking speed: distance traveled in a given time in meters/second, if cadence is in steps per minute and stride length is in meters the calculation is given by:
    \begin{equation}
        \frac{cadence*stride\hspace{1mm}length}{120}
    \end{equation}
\end{itemize}

Gait is globally divided into two phases, \textit{stance} and \textit{swing}. The first occur when foot is in contact with the ground, on the other hand, the swing occur when it is not. The stance phase ends when foot off (toe off); the same point at which the swing begins, as illustrated in the figure \ref{gait_phases}, 

\begin{figure}
    \centering
    \includegraphics[width=1\textwidth]{gait_phases.png}
    \caption[One side gait phases]{One side gait phases. Source: edited by author from \cite{Baker2013}.}
    \label{gait_phases}
\end{figure}

The scheme can include the events of both legs, and is divided into first double support (both feet in contact with ground), single support, second double support and swing, as is shown in figure \ref{gait_phases_ds}. Single support and swing are long phases, so a subdivision is a good choice: \textit{early}, \textit{middle} and \textit{late} as can be seen in the figure \cite{Baker2013, schneck2002biomechanics}.

\begin{figure}[b]
    \centering
    \includegraphics[width=1\textwidth]{gait_phases_ds.png}
    \caption[Both sides gait phases ]{Both sides gait phases. Source: edited by author from \cite{Baker2013}.}
    \label{gait_phases_ds}
\end{figure}

Any impairment in the ankle joint manifests as alterations in gait parameters, disrupting the cyclical and repetitive nature of walking. 
This leads to increased energy expenditure and triggers compensatory mechanisms in other muscle groups to maintain locomotion, often resulting
in instability and a heightened risk of falling.

The PRANK ankle rehabilitation robot is grounded in biomechanical, neurophysiological, and human-machine interaction principles 
that guide its design and control strategy. Below outlines the theoretical foundations that justify the system's architecture, 
control approach, and therapeutic goals.

\section{Biomechanical Modeling of the Ankle}
The human ankle plays a critical role in gait, contributing to propulsion, balance, and shock absorption. 
It operates in coordination with the knee and hip as part of the lower limb kinematic chain. 
Biomechanical models often simplify the ankle as a single-degree-of-freedom joint, focusing on dorsiflexion and plantarflexion, 
while more advanced models incorporate multi-axis dynamics and impedance variations across gait phases \cite{opensim2024}. 
These models are essential for simulating joint behavior and designing control strategies that replicate natural movement.

\section{Motor Rehabilitation and Neuroplasticity}
Motor recovery after injury or neurological impairment relies on neuroplasticity—the brain's ability to reorganize through repetitive, 
goal-oriented movement. Rehabilitation robotics leverage this principle by enabling high-intensity, task-specific training. 
Key motor learning principles include feedback, variability, and active participation, all of which enhance cortical reorganization and 
functional recovery \cite{physiopedia2024,neuability2024}. The PRANK system integrates these principles through interactive tasks and 
adaptive control.

\section{Human-Machine Interfaces in Rehabilitation}
Effective rehabilitation requires intuitive and responsive interfaces between the user and the robotic system. 
These interfaces must accommodate physical comfort, cognitive engagement, and safety. Advances in flexible electronics and soft 
robotics have enabled more ergonomic designs that conform to the user's anatomy \cite{gao2021flexible}. Additionally, 
virtual reality (VR) environments provide immersive feedback, increasing motivation and facilitating motor learning through 
gamified rehabilitation tasks \cite{frontiers2021interfaces}.

\section{Clinical Evaluation Metrics}
Quantifying rehabilitation outcomes is essential for validating system effectiveness. Clinical metrics include range of motion, 
gait symmetry, balance, and patient-reported outcomes. Instruments such as the Rehabilitation Measures Database offer standardized 
tools for assessing ankle functionality and tracking progress over time \cite{sralab2024}. These metrics inform both the design of 
therapeutic protocols and the evaluation of system performance.



