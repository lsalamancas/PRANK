\chapter{PROJECT JUSTIFICATION} \label{results}
\graphicspath{{images/}}

Gait is a fundamental aspect of human independence and well-being. Within the gait kinetics, the ankle joint plays a pivotal role in absorbing impact forces during heel strike and generating propulsion during toe-off. When the ankle's function is impaired—due to conditions such as stroke, orthopedic trauma, or neurodegenerative disease—patients face significant limitations in mobility, balance, and daily autonomy.

Rehabilitation is essential for restoring ankle functionality, yet conventional methods often lack the consistency, objectivity, and sustained engagement required for optimal recovery. Furthermore, existing robotic devices for gait rehabilitation typically do not include mechanisms specifically designed for targeted ankle strength recovery. In this context, robotic rehabilitation systems offer the advantages of repeatability, precise measurement, and the ability to adapt therapy to an individual’s progress. Additionally, serious games and virtual reality environments have shown considerable promise in enhancing motivation and participation—key factors that directly influence rehabilitation outcomes.

This project addresses these needs through the development of PRANK (Parallel Robot for ANKle rehabilitation), a parallel robotic platform actuated by an RRR configuration and controlled through admittance strategies. It aims to offer a safe, adaptive, and engaging therapeutic environment. By integrating a serious game and a real-time feedback loop that adjusts difficulty based on force and range of motion (ROM), the system seeks to promote user-centered recovery while providing measurable, high-fidelity data. Additionally, validating PRANK’s ROM measurement capabilities against an optoelectronic motion capture system will support its use as a reliable biomechanical assessment tool in future clinical settings.