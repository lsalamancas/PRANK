%\null \vspace{\fill}

\section*{Abstract}
\phantomsection
\addcontentsline{toc}{section}{\numberline{}Abstract}
\graphicspath{{images/}}


Human gait is a fundamental component of mobility, autonomy, and overall quality of life. Among the joints involved in locomotion, the ankle plays a crucial role in absorbing shock during heel strike and facilitating propulsion during the push-off phase. Impairments in ankle function—whether due to neurological conditions, trauma, or post-surgical recovery—can significantly compromise gait, balance, and independence. Robotic rehabilitation has emerged as a promising alternative to traditional therapy, offering consistent, repeatable, and quantitatively measurable support.

This project proposes the development of a Parallel Ankle Rehabilitation Robot (PARR), actuated through an RRR electric motor configuration and controlled via an admittance algorithm based on real-time force feedback. The system is complemented by a serious game implemented in virtual reality, designed to enhance patient engagement and tailor therapy difficulty dynamically based on ankle range of motion (ROM) and user-applied force. The project focuses on the mechatronic development of the robot and the comparative validation of its ROM measurement accuracy against an optoelectronic motion capture system, aiming to demonstrate the system’s feasibility for reliable biomechanical evaluation and future deployment in clinical environments.


